\documentclass[preprint,12pt,authoryear]{elsarticle}
%\documentclass[final,1p,times,twocolumn,authoryear]{elsarticle}
\usepackage{lineno,hyperref}
\modulolinenumbers[5]

\journal{Journal of \LaTeX\ Templates}

%%%%%%%%%%%%%%%%%%%%%%%
%% Elsevier bibliography styles
%%%%%%%%%%%%%%%%%%%%%%%
%% To change the style, put a % in front of the second line of the current style and
%% remove the % from the second line of the style you would like to use.
%%%%%%%%%%%%%%%%%%%%%%%

%% Numbered
%\bibliographystyle{model1-num-names}

%% Numbered without titles
%\bibliographystyle{model1a-num-names}

%% Harvard
\bibliographystyle{model2-names.bst}\biboptions{authoryear}

%% Vancouver numbered
%\usepackage{numcompress}\bibliographystyle{model3-num-names}

%% Vancouver name/year
%\usepackage{numcompress}\bibliographystyle{model4-names}\biboptions{authoryear}

%% APA style
%\bibliographystyle{model5-names}\biboptions{authoryear}

%% AMA style
%\usepackage{numcompress}\bibliographystyle{model6-num-names}

%% `Elsevier LaTeX' style
%\bibliographystyle{elsarticle-harv}
%%%%%%%%%%%%%%%%%%%%%%%

\begin{document}

\begin{frontmatter}

\title{Analysis of geologic maps to support soil survey - A case study from South Tyrol}


%% Group authors per affiliation:
\author{Fabian E. Gruber\fnref{myfootnote}}
\author{Jasmin Baruck\fnref{myfootnote}}
\author{Clemens Geitner\fnref{myfootnote}}

\address{University of Innsbruck}

\begin{abstract}

\end{abstract}

\begin{keyword}

\end{keyword}

\end{frontmatter}

\linenumbers

\section{Introduction}
\paragraph{general introduction}
Geologic maps have always been an important aid in soil survey as parent material is an important factor in soil formation \citep{Jenny1941}. The importance of this relationship is highlighted by the fact that, vice versa, soil maps have themselves been applied to support and improve geologic mapping \citep{Brevik2015}. 

\paragraph{overview of literature with regard to the use of geologic maps in classic field soil survey and digital soil mapping in literature}
\paragraph{literature overview with regard to terrain parameters for characterisation and soil survey}
\paragraph{overview of intention and aims}
In a first step we analyse how well the geologic units of the high resolution geologic map correspond to the parent material identified by the soil surveyor. This requires generalisation of the geologic units into categories that can be compared to the parent material units used in the soil (or forestry) surveys. The result is a confusion matrix that shows to which extent geologic units are in accordance with the parent material mapped by the surveyor.  We highlight those units that are often confused or show overlap, and which should consequently be surveyed  

Using a data mining approach based on a forward stepwise feature selection with a SVM classifier, we then identify which terrain parameters best separate geologic units and how they can be related to and interpreted with regard to soil formation and the distribution of soil units.

The connection between the two important soil forming factors, parent material and topography, on the one hand, and soil as the result of theses factors on the other, is then investigated by analysing the occurence and distribution of soils for each geologic unit. A sythesis of this information then leads to a Steckbier that characterises each geologic unit. 

The aim of this study is to evaluate the performance and applicability of geologic maps to support soil survey in South Tyrol and how to make best use of this information. Hence each geologic unit is characterised with regard to topography and soil and we highlight those units were there es often dissent between soil parent material as mapped by the soil surveyor and the geologic units mapped by geologists.

 We propose that future surveys focus on these units with increased uncertainty with regard to soil parent material. The Steckbriefe shall give the surveyor an overview of which geologic units require special attention due to their diversity regarding topography and soils ...hier steh ich an, der Absatz mit den Zielen muss noch besser werden. Performing such a procedure is advised in advance of new detailed soil surveys to make best use of available information and concentrate the time and money consuming task of field soil survey, involving soil pits and auguring, on units identified as   highly variable and uncertain regarding soils?? 




\section{Material and Methods}
\subsection{Study area and data}
\paragraph{Overall description}
\subsubsection{Geology}

\begin{table}[ht]
\centering
\begin{tabular}{p{4.5cm}cp{6cm}}
  \hline
geounit & Abbrev. & short description \\ 
  \hline
 \raisebox{-1.5ex}{alluvial deposits} & \raisebox{-1.5ex}{AD} & recent and pleistocenic deposits of silt, sand and gravels \\ 
colluvial deposits & CD & footslope deposits \\ 
calcareous sedimentary rock & \raisebox{-1.5ex}{CSR} & \raisebox{-1.5ex}{limestones and dolomites} \\ 
\raisebox{-1.5ex}{debris cones} & \raisebox{-1.5ex}{DC} & recent conic deposits from debris flows and gulleys \\ 
glaciolacustrine deposits & GD & (fine) sand deposits with dropstones \\ 
ice-marginal sediments & IMS & clast-supported gravels \\ 
intermediate sedimentary rock & \raisebox{-1.5ex}{ISR} & \raisebox{-1.5ex}{silt- and sandstones} \\ 
\raisebox{-1.5ex}{mire deposits} &\raisebox{-1.5ex}{MD} & recent and pleistocenic silt and peat deposits \\ 
\raisebox{-1.5ex}{mixed deposits} & \raisebox{-1.5ex}{MD} & pleistocenic deposits from debris flows and gulleys \\ 
\raisebox{-1.5ex}{siliceous bedrock} & \raisebox{-1.5ex}{SB} & rhyolite and rhyodazite tuffs and ignimbrites \\ 
Scree & SC & recent and pleistocenic blocky deposits \\ 
\raisebox{-1.5ex}{slope debris} & \raisebox{-1.5ex}{SD} & recent and pleistocen debris on slopes \\ 
sub-glacial till & SGT & compacted sub-glacial sediment \\ 
\raisebox{-0ex}{silicious sedimentary rock} & \raisebox{-0ex}{SSR} & sandstones and siltstones \\ 
till in general & TG & undifferentiated glacial sediment \\ 
   \hline
\end{tabular}
\caption{Table of the generalised parent material geounits with abbreviations and a short description.} 
\label{geounits}
\end{table}

\subsubsection{Soils}
\subsection{Terrain parameters with emphasis on roughness measures}
\citep{Riley1999}
\section{Results}
\subsection{Comparison of soil parent material at soil profile sites with geologic map units}
\begin{table}[ht]
\centering
\tabcolsep=0.10cm
\begin{tabular}{cccccccccccccccc}
  \hline
 & AD & CD & CSR & GD & IMS & ISR & LD & MC & MD & SB & SC & SD & SGT & SSR & TG \\ 
  \hline
AD &   7 &   0 &   0 &   0 &   1 &   0 &   0 &   4 &   0 &   0 &   0 &   0 &   1 &   0 &   0 \\ 
  CD &   0 &   3 &   1 &   0 &   0 &   0 &   0 &   3 &   1 &   1 &   0 &   0 &   0 &   0 &   0 \\ 
  CSR &   0 &   0 &   2 &   0 &   0 &   0 &   0 &   0 &   0 &   0 &   0 &   2 &   0 &   0 &   1 \\ 
  GD &   1 &   0 &   0 &   5 &   0 &   0 &   0 &   0 &   2 &   2 &   0 &   0 &   1 &   0 &   0 \\ 
  IMS &   0 &   0 &   0 &   0 &   0 &   0 &   0 &   0 &   0 &   0 &   0 &   0 &   0 &   0 &   0 \\ 
  ISR &   0 &   1 &   7 &   0 &   0 &   0 &   0 &   2 &   0 &   1 &   0 &   4 &   0 &   2 &   3 \\ 
  LD &   0 &   0 &   0 &   0 &   0 &   0 &   0 &   0 &   0 &   0 &   0 &   0 &   0 &   0 &   0 \\ 
  MC &   0 &   0 &   0 &   0 &   0 &   0 &   0 &   5 &   0 &   0 &   0 &   1 &   0 &   0 &   0 \\ 
  MD &   0 &   0 &   1 &   0 &   0 &   0 &   0 &   0 &   0 &   2 &   0 &   0 &   3 &   1 &   3 \\ 
  SB &   0 &   0 &   0 &   0 &   0 &   0 &   0 &   0 &   0 &  14 &   0 &   4 &   0 &   0 &   0 \\ 
  SC &   0 &   0 &   0 &   0 &   0 &   0 &   0 &   0 &   0 &   1 &   4 &   3 &   0 &   0 &   0 \\ 
  SD &   1 &   1 &  13 &   0 &   0 &   3 &   0 &  20 &   6 &   3 &   8 &  55 &   1 &   3 &   8 \\ 
  SGT &   0 &   0 &   0 &   0 &   0 &   0 &   0 &   0 &   0 &   0 &   0 &   0 &   0 &   0 &   0 \\ 
  SSR &   0 &   0 &   0 &   0 &   0 &   0 &   0 &   0 &   0 &   1 &   0 &   0 &   0 &   3 &   0 \\ 
  TG &   3 &   0 &  12 &   2 &   0 &   0 &   0 &   2 &   2 &  24 &   2 &  15 &  40 &   1 &  48 \\ 
   \hline
\end{tabular}
\caption{Tabular comparison of parent material geounits as observed by soil surveyor (rows) and in the geologic map (columns).} 
\label{kartiergegenkarte}
\end{table}
\subsection{Geomorphometric analysis of geologic map units}
\subsection{Distribution of soils with regard to geologic units}
\section{Discussion}
\subsection{Differences in what is mapped}
Between the two different frameworks of mapping, geology on the one hand and soil on the other, it is important to acknowledge the main focus of attention of each branch of research. There may exist a difference with regard to how pronounced a certain feature or characteristic must be in order to considered for mapping. 

A typical example is..

\section*{References}
\bibliography{P2.bib}

\end{document}